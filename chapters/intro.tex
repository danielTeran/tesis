-Introducci�n 

A lo largo de los a�os se han generado una cantidad enorme de informaci�n de car�cter biol�gico y medico que se requiere tener al alcance de los cientificos para encontrar nuevas interacciones



*Comparative genomics
Comparative genomics is the study of the relationship of genome structure and function across different biological
species. Gene finding is an important application of comparative genomics, as is discovery of new, non-coding
functional elements of the genome. Comparative genomics exploits both similarities and differences in the proteins,
RNA, and regulatory regions of different organisms. Computational approaches to genome comparison have
recently become a common research topic in computer science. 
Analysis of mutations in cancer 
In cancer, the genomes of affected cells are rearranged in complex or even unpredictable ways. Massive sequencing
efforts are used to identify previously unknown point mutations in a variety of genes in cancer. Bioinformaticians
continue to produce specialized automated systems to manage the sheer volume of sequence data produced, and they
create new algorithms and software to compare the sequencing results to the growing collection of human genome
sequences and germline polymorphisms. New physical detection technologies are employed, such as oligonucleotide
microarrays to identify chromosomal gains and losses and single-nucleotide polymorphism arrays to detect known
point mutations. Another type of data that requires novel informatics development is the analysis of lesions found to
be recurrent among many tumors. *

-Redes de regulaci�n transcripciones 

-importancia en biolog�a y biomedicina. 



-Miner�a de datos en bioinform�tica y biolog�a computacional


-Teor�a de grafos y redes complejas